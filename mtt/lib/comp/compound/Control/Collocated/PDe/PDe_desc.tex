% -*-latex-*- Put EMACS into LaTeX-mode
% Verbal description for system PDe (PDe_desc.tex)
% Generated by MTT on Tue May 1 09:26:33 BST 2001.

% %%%%%%%%%%%%%%%%%%%%%%%%%%%%%%%%%%%%%%%%%%%%%%%%%%%%%%%%%%%%%%%
% %% Version control history
% %%%%%%%%%%%%%%%%%%%%%%%%%%%%%%%%%%%%%%%%%%%%%%%%%%%%%%%%%%%%%%%
% %% $Id$
% %% $Log$
% %% Revision 1.1  2000/12/28 09:13:38  peterg
% %% Initial revision
% %%
% %%%%%%%%%%%%%%%%%%%%%%%%%%%%%%%%%%%%%%%%%%%%%%%%%%%%%%%%%%%%%%%

   The acausal bond graph of system \textbf{PDe} is
   displayed in Figure \Ref{fig:PDe_abg.ps} and its label
   file is listed in Section \Ref{sec:PDe_lbl}.
   The subsystems are listed in Section \Ref{sec:PDe_sub}.

   This is a proportional + derivative (PD) controller for a
   collocated sutuation where the control signal is an effort and the
   measured signal is a (collocated) flow.

   The controller can be thought of as controlling \emph{integated
   flow}, and it is with respect to this that the P and D terms are defined.

 The setpoint is a \emph{flow}; and must be generated to give the
 desired \emph{integrated} flow.

 Physically, the controller is a \textbf{C} and an \textbf{R}
 component - for mechanical systems a mass and a spring.
 
 Mathematically, in integral causality, the equations are:
%file: pde_{dae}.tex
%differential-algebraic equations
 \begin{equation}
   \begin{aligned}
     \dot x_{1} &=
     {
       f_d - f
       }
   \end{aligned}
 \end{equation}
 \begin{equation}
   \begin{aligned}
     u &=
     {
       - k_{d} f + k_{p} x_{1}
       }
   \end{aligned}
 \end{equation}

 The state $x_1$ is the the integrated difference between 
 \emph{desired} flow $f_d$ and the actual flow  $f$. Thus the control
 signal $u$ is $k_p$ multiplied by the position error minus $k_d$ time
 the flow.