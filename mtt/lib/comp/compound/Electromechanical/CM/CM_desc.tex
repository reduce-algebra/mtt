% Verbal description for system CM (CM_desc.tex)
% Generated by MTT on Fri Sep 19 18:07:12 BST 1997.

% %%%%%%%%%%%%%%%%%%%%%%%%%%%%%%%%%%%%%%%%%%%%%%%%%%%%%%%%%%%%%%%
% %% Version control history
% %%%%%%%%%%%%%%%%%%%%%%%%%%%%%%%%%%%%%%%%%%%%%%%%%%%%%%%%%%%%%%%
% %% $Id$
% %% $Log$
% %%%%%%%%%%%%%%%%%%%%%%%%%%%%%%%%%%%%%%%%%%%%%%%%%%%%%%%%%%%%%%%

   The acausal bond graph of system \textbf{CM} is
   displayed in Figure \Ref{CM_abg} and its label
   file is listed in Section \Ref{sec:CM_lbl}.
   The subsystems are listed in Section \Ref{sec:CM_sub}.

{\bf CM} is an electromechanical moving-plate  capacitor with linear
electrical capacitance $c$ of the form
\begin{equation}
  c = c_0 \frac{x_0}{x}
\end{equation}
where $x_0$ is the plate separation corresponding to a capacitance of
$c_0$.
The corresponding electrical constitutive relationship (which gives an
energy-conserving two-port \textbf{C}) is
\begin{equation}
  F = Q c_0 \frac{x_0}{x^2}
\end{equation}
where $F$ is the force between the plates and $Q$ the charge on the
capacitor.
This is implemented in the \emph{cm.cr} Constitutive Relationship.
