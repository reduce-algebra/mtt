% -*-latex-*- Put EMACS into LaTeX-mode
% Verbal description for system BernoulliEuler (BernoulliEuler_desc.tex)
% Generated by MTT on Mon Apr 19 06:43:36 BST 1999.

% %%%%%%%%%%%%%%%%%%%%%%%%%%%%%%%%%%%%%%%%%%%%%%%%%%%%%%%%%%%%%%%
% %% Version control history
% %%%%%%%%%%%%%%%%%%%%%%%%%%%%%%%%%%%%%%%%%%%%%%%%%%%%%%%%%%%%%%%
% %% $Id$
% %% $Log$
% %%%%%%%%%%%%%%%%%%%%%%%%%%%%%%%%%%%%%%%%%%%%%%%%%%%%%%%%%%%%%%%

   The acausal bond graph of system \textbf{BernoulliEuler} is
   displayed in Figure \Ref{fig:BernoulliEuler_abg.ps} and its label
   file is listed in Section \Ref{sec:BernoulliEuler_lbl}.
   The subsystems are listed in Section \Ref{sec:BernoulliEuler_sub}.

This component represents one lump of a lumped model of a uniform beam
modelled using the the Bernoulli-Euler assumptions:
\begin{enumerate}
\item The shear forces can be neglected.
\item Rotational inertia can be neglected.
\end{enumerate}

\begin{itemize}
\item The \textbf{I} component represents the inertial properties of
  the lump in the perpendicular direction. In particular the velocity
  of the lump $v$ is:
  \begin{equation}
    \dot v = \frac{\Delta f}{\Delta m}
  \end{equation}
  where $\Delta m$ is the lump mass and $\Delta f$ is the net vertical
  force.
\item The \textbf{C} component represents the angular stiffness of the
  lump. In particular the torque acting on the lump is:
  \begin{equation}
    \dot \tau =  \Delta k \Delta \Omega
  \end{equation}
  where $\Delta k$ is the lump (angular) stiffness and $\Delta \Omega$
  is the net angular velocity.
\item The \textbf{TF} component represents the relation between the
  angular domains
  \begin{equation}
    \begin{align}
      \tau &= \Delta x \Delta f \\
      \Delta v &= \Delta x \Omega 
    \end{align}
  \end{equation}
\end{itemize}
