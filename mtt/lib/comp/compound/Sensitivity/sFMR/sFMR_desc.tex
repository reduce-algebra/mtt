% -*-latex-*- Put EMACS into LaTeX-mode
% Verbal description for system sFMR (sFMR_desc.tex)
% Generated by MTT on Thu Jul 5 23:47:35 BST 2001.

% %%%%%%%%%%%%%%%%%%%%%%%%%%%%%%%%%%%%%%%%%%%%%%%%%%%%%%%%%%%%%%%
% %% Version control history
% %%%%%%%%%%%%%%%%%%%%%%%%%%%%%%%%%%%%%%%%%%%%%%%%%%%%%%%%%%%%%%%
% %% $Id$
% %% $Log$
% %% Revision 1.1  2000/12/28 09:13:38  peterg
% %% Initial revision
% %%
% %%%%%%%%%%%%%%%%%%%%%%%%%%%%%%%%%%%%%%%%%%%%%%%%%%%%%%%%%%%%%%%

   The acausal bond graph of system \textbf{sFMR} is
   displayed in Figure \Ref{fig:sFMR_abg.ps} and its label
   file is listed in Section \Ref{sec:sFMR_lbl}.
   The subsystems are listed in Section \Ref{sec:sFMR_sub}.


This is the sensitivity version of the \textbf{FMR} (flow-modulated
resistor) component.

In the linear case, the CR of the standard port is:
\begin{equation}
  e = mrf
\end{equation}
where $e$ is the effort, $m$ the (flow) modulation, $r$ the
``resistance'' and $f$ the flow.

The corresponding sensitivity CR is:
\begin{equation}
  e^\prime  = m^\prime rf + mr^\prime f + mrf^\prime
\end{equation}