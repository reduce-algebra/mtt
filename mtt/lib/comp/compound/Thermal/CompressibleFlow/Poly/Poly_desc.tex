% -*-latex-*- Put EMACS into LaTeX-mode
% Verbal description for system Poly (Poly_desc.tex)
% Generated by MTT on Thu Mar 19 13:24:59 GMT 1998.

% %%%%%%%%%%%%%%%%%%%%%%%%%%%%%%%%%%%%%%%%%%%%%%%%%%%%%%%%%%%%%%%
% %% Version control history
% %%%%%%%%%%%%%%%%%%%%%%%%%%%%%%%%%%%%%%%%%%%%%%%%%%%%%%%%%%%%%%%
% %% $Id$
% %% $Log$
% %% Revision 1.3  1998/03/31 15:05:33  peterg
% %% Spell checked
% %%
% %% Revision 1.2  1998/03/27 10:56:14  peterg
% %% Added bicausal bit
% %%
% %% Revision 1.1  1998/03/26 15:13:35  peterg
% %% Initial revision
% %%
% %%%%%%%%%%%%%%%%%%%%%%%%%%%%%%%%%%%%%%%%%%%%%%%%%%%%%%%%%%%%%%%

   The acausal bond graph of system \textbf{Poly} is
   displayed in Figure \Ref{Poly_abg} and its label
   file is listed in Section \Ref{sec:Poly_lbl}.

This four-port component computes the temperature following a
polytropic expansion using:
\begin{equation}
  T_2 = T_1 \left ( \frac{P_2}{P_1} \right )^\alpha
\end{equation}
where $\alpha = \frac{n-1}{n}$ and $n$ is the coefficient of
polytropic expansion.  This component imposes zero flow at all its
ports and therefore does not affect energy balance.

The output is \emph{bicausal} as it imposes both $T_2$ and a zero flow.
This is implemented using the bicausal \textbf{SS} component labeled
``zero''.

