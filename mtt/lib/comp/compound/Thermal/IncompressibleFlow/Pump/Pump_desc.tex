% Verbal description for system Pump (Pump_desc.tex)
% Generated by MTT on Mon Mar 9 09:11:22 GMT 1998.

% %%%%%%%%%%%%%%%%%%%%%%%%%%%%%%%%%%%%%%%%%%%%%%%%%%%%%%%%%%%%%%%
% %% Version control history
% %%%%%%%%%%%%%%%%%%%%%%%%%%%%%%%%%%%%%%%%%%%%%%%%%%%%%%%%%%%%%%%
% %% $Id$
% %% $Log$
% %%%%%%%%%%%%%%%%%%%%%%%%%%%%%%%%%%%%%%%%%%%%%%%%%%%%%%%%%%%%%%%

   The acausal bond graph of system \textbf{Pump} is
   displayed in Figure \Ref{Pump_abg} and its label
   file is listed in Section \Ref{sec:Pump_lbl}.
   The subsystems are listed in Section \Ref{sec:Pump_sub}.

\textbf{Pump} represents an ideal pump for incompressible fluid
driving fluid though a \textbf{Pipe} component. The pipe component
provides the correct thermal flow; if its resistance is set to zero,
the pump is an ideal component.

The flow must be one way (in to out) for correct thermal properties.

The ports are 
\begin{itemize}
\item [Hy_in] Pressure/volume-flow inflow
\item [Hy_in] Pressure/volume-flow outflow
\item [Th_in] Temperature/Entropy-flow in flow
\item [Th_out] Temperature/Entropy-flow out flow
\item [Shaft] Torque/angular velocity input.
\end{itemize}

%%% Local Variables: 
%%% mode: latex
%%% TeX-master: t
%%% End: 
