% -*-latex-*- Put EMACS into LaTeX-mode
% Verbal description for system Reactor (Reactor_desc.tex)
% Generated by MTT on Fri Mar 3 12:43:33 GMT 2000.

% %%%%%%%%%%%%%%%%%%%%%%%%%%%%%%%%%%%%%%%%%%%%%%%%%%%%%%%%%%%%%%%
% %% Version control history
% %%%%%%%%%%%%%%%%%%%%%%%%%%%%%%%%%%%%%%%%%%%%%%%%%%%%%%%%%%%%%%%
% %% $Id$
% %% $Log$
% %%%%%%%%%%%%%%%%%%%%%%%%%%%%%%%%%%%%%%%%%%%%%%%%%%%%%%%%%%%%%%%

\fig{Reactor_pic}
{Reactor_pic} {0.9} {System \textbf{Reactor}, Schematic}

Figure \Ref{fig:Reactor_pic} is the schematic diagram od a chemical reactor.
The acausal bond graph of system \textbf{Reactor} is displayed in
Figure \Ref{fig:Reactor_abg.ps} and its label file is listed in
Section \Ref{sec:Reactor_lbl}.  The subsystems are listed in Section
\Ref{sec:Reactor_sub}.

This example of a (nonlinear) chemical reactor is due to Trickett and
Bogle\footnote{ K. J. Tricket, \emph{Quantification of Inverse
    Responses for Controllability Assessment of Nonlinear Processes},
  PhD Thesis, University College London, 1994} is used in this
section.  The reactor has two reaction mechanisms: $\text{A}
\rightarrow \text{B} \rightarrow \text{C}$ and $\text{2A} \rightarrow
\text{D}$.  The reactor mass inflow and outflow $f_r$ are identical.
$q$ represents the heat inflow to the reactor.

This is a two input, two-output unstable nonlinear system with unstable zero
dynamics.
The following figures illustrate the properties of the
\emph{linearised} system.

\fig{Reactor_pole_1_2}
{Reactor_pole_1_2} {0.9} {System \textbf{Reactor}: poles 1 and 2
  v. steady-state flow $f_s$}

\fig{Reactor_pole_3}
{Reactor_pole_3} {0.9} {System \textbf{Reactor}: pole 3
  v. steady-state flow $f_s$}

\fig{Reactor_zero_a}
{Reactor_zero_a} {0.9} {System \textbf{Reactor}: zero of system with
  $t$ and $c_a$ as output
  v. steady-state flow $f_s$}

\fig{Reactor_zero_b}
{Reactor_zero_b} {0.9} {System \textbf{Reactor}: pole 3
  v. steady-state flow $f_s$}

\begin{itemize}
\item Figures \Ref{fig:Reactor_pole_1_2} and
  \Ref{fig:Reactor_pole_3} show the three poles of the
  \emph{linearised} system as the steady-state flow varies. 
\item Figure \Ref{fig:Reactor_zero_a} shows the system zero (when $t$ and
  $c_a$ are the two system outputs) as the
  \emph{linearised} system as the steady-state flow varies. 
\item Figure \Ref{fig:Reactor_zero_b} shows the system zero (when $t$ and
  $c_b$ are the two system outputs) as the
  \emph{linearised} system as the steady-state flow varies. 
\end{itemize}
