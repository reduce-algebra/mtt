% -*-latex-*- Put EMACS into LaTeX-mode
% Verbal description for system ReactorTF (ReactorTF_desc.tex)
% Generated by MTT on Fri Mar 3 12:43:33 GMT 2000.

% %%%%%%%%%%%%%%%%%%%%%%%%%%%%%%%%%%%%%%%%%%%%%%%%%%%%%%%%%%%%%%%
% %% Version control history
% %%%%%%%%%%%%%%%%%%%%%%%%%%%%%%%%%%%%%%%%%%%%%%%%%%%%%%%%%%%%%%%
% %% $Id$
% %% $Log$
% %%%%%%%%%%%%%%%%%%%%%%%%%%%%%%%%%%%%%%%%%%%%%%%%%%%%%%%%%%%%%%%

\fig{ReactorTF_pic}
{ReactorTF_pic} {0.9} {System \textbf{ReactorTF}, Schematic}

Figure \Ref{fig:ReactorTF_pic} is the schematic diagram of a chemical
reactor.

The acausal bond graph of system \textbf{ReactorTF} is displayed in
Figure \Ref{fig:ReactorTF_abg.ps} and its label file is listed in
Section \Ref{sec:ReactorTF_lbl}.  The subsystems are listed in Section
\Ref{sec:ReactorTF_sub}.

This example of a (nonlinear) chemical reactor is due to Trickett and
Bogle\footnote{ K. J. Tricket, \emph{Quantification of Inverse
    Responses for Controllability Assessment of Nonlinear Processes},
  PhD Thesis, University College London, 1994} is used in this
section.  The reactor has two reaction mechanisms: $\text{A}
\rightarrow \text{B} \rightarrow \text{C}$ and $\text{2A} \rightarrow
\text{D}$.  The reactor mass inflow and outflow $f_r$ are identical.
$q$ represents the heat inflow to the reactor.

The control loop $t$/$f$ has been inverted. The resulting SISO
system has two interpretations:
\begin{enumerate}
\item the \emph{dynamics} of the $c_b$/$q$ loop when the $t$/$f$ loop
  is under perfect control and
\item the \emph{inverse} dynamics of the  $t$/$f$ loop.
\end{enumerate}

\fig{ReactorTF_zero_1} {ReactorTF_zero_1} {0.9}
{System\textbf{ReactorTF}: zero 1 v flow} 
\fig{ReactorTF_zero_2} {ReactorTF_zero_2} {0.9}
{System\textbf{ReactorTF}: zero 2 v flow} 

Figures \Ref{fig:ReactorTF_zero_1} and \Ref{fig:ReactorTF_zero_2}
shows the poles of the linearised system as the steady-state flow
varies: these are the \emph{zeros} of the $t$/$f$ control-loop when
the $c_b$/$q$ loop is \emph{open}.


