% -*-latex-*- Put EMACS into LaTeX-mode
% Verbal description for system PPPCantileverBeam (PPPCantileverBeam_desc.tex)
% Generated by MTT on Mon Apr 19 07:04:54 BST 1999.

% %%%%%%%%%%%%%%%%%%%%%%%%%%%%%%%%%%%%%%%%%%%%%%%%%%%%%%%%%%%%%%%
% %% Version control history
% %%%%%%%%%%%%%%%%%%%%%%%%%%%%%%%%%%%%%%%%%%%%%%%%%%%%%%%%%%%%%%%
% %% $Id$
% %% $Log$
% %% Revision 1.1  1999/05/18 04:01:50  peterg
% %% Initial revision
% %%
% %%%%%%%%%%%%%%%%%%%%%%%%%%%%%%%%%%%%%%%%%%%%%%%%%%%%%%%%%%%%%%%

The acausal bond graph of system \textbf{PPPCantileverBeam} is displayed in
Figure \Ref{fig:PPPCantileverBeam_abg.ps} and its label file is listed in
Section \Ref{sec:PPPCantileverBeam_lbl}.  The subsystems are listed in Section
\Ref{sec:PPPCantileverBeam_sub}.
   
This example represents the dynamics of a uniform beam with one fixed
and one free end.  The beam is approximated by 16 equal lumps using
the Bernoulli-Euler approximation with damping. The ouputs are taken
to be the 16 lump velocities, the two inputs are taken to be torques
applied to lumps 3 away from each end; this approximates the effect of
two piezoelectric patches.

The system parameters are given in Section
\Ref{sec:PPPCantileverBeam_numpar.tex}.
The system has 32 states (16 modes of vibration), 2 inputs and 12
outputs.
The first 8 modal frequencies are given in Table \ref{tab:modes}.
\begin{table}[htbp]
  \begin{center}
    \begin{tabular}{||c|c||c|c||}
      \hline
      Mode & $\omega_i$ & Mode & $\omega_i$\\
      \hline
      \hline
      1 & 0.0079504 & 5 & 0.4352376\\
      2 & 0.0498140 & 6 & 0.6338899\\
      3 & 0.1386209 & 7 & 0.8573604\\
      4 & 0.2682835 & 8 & 1.0976942\\
      \hline
    \end{tabular}
    \caption{Modal frequencies}
    \label{tab:modes}
  \end{center}
\end{table}

Figure \Ref{fig:PPPCantileverBeam_lmfr.ps} shows the log modulus of the frequency
response and Figure \Ref{fig:PPPCantileverBeam_pppy0.ps} shows the system transient
reponse with no control and an initial unit twist.

Figure \Ref{fig:PPPCantileverBeam_pppy.ps} shows closed loop reponse when
controlled by a PPP controller attempting to damp modes 1 and 2 from
input 1 and modes 1 and three from input 2. Figure
\Ref{fig:PPPCantileverBeam_pppy.ps} shows the corresponding control signals.







