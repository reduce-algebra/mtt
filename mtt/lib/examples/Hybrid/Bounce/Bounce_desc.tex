% Verbal description for system Bounce (Bounce_desc.tex)
% Generated by MTT on Wed Jul 2 10:55:51 BST 1997.

% %%%%%%%%%%%%%%%%%%%%%%%%%%%%%%%%%%%%%%%%%%%%%%%%%%%%%%%%%%%%%%%
% %% Version control history
% %%%%%%%%%%%%%%%%%%%%%%%%%%%%%%%%%%%%%%%%%%%%%%%%%%%%%%%%%%%%%%%
% %% $Id$
% %% $Log$
% %% Revision 1.1  2000/12/28 17:45:24  peterg
% %% To RCS
% %%
% %%%%%%%%%%%%%%%%%%%%%%%%%%%%%%%%%%%%%%%%%%%%%%%%%%%%%%%%%%%%%%%

   The acausal bond graph of system \textbf{Bounce}, togehter with a
   schematic diagram is
   displayed in Figure \Ref{Bounce_abg} and its label
   file is listed in Section \Ref{sec:Bounce_lbl}.
   The subsystems are listed in Section \Ref{sec:Bounce_sub}.

The model uses the {\bf CSW} switched {\bf C} element to simulate
contact with the ground. The corresponding switching function (See
Section \ref{sec:Bounce_input-noargs.txt}), is based on the height above the
ground $h$ as follows:
\begin{equation}
i_{sw} = 
  \begin{cases}
    0 & \text{if $h > 0$}\\
    -1 & \text{if $h \le 0$}
  \end{cases}
\end{equation}

In other words, the component acts as an ideal spring when the ball is
in contact with the ground yet has no effect when the ball is not in
contact with the ground.

The ball is modelled as a point mass (the \textbf{I} component) and  a
linear resistance to motion (the  (the \textbf{R} component).

The system was simulated for 100 time units and the resultant height
is plotted in Figure \ref{fig:Bounce_odeso-noargs.ps}. The ball was released at zero
velocity from a height of ten units. The bounce height decreases due to
the effect of the modelled air resistance.

%%% Local Variables: 
%%% mode: latex
%%% TeX-master: t
%%% End: 
