% -*-latex-*- Put EMACS into LaTeX-mode
% Verbal description for system idRC (idRC_desc.tex)
% Generated by MTT on Thu Apr 5 11:04:33 BST 2001.

% %%%%%%%%%%%%%%%%%%%%%%%%%%%%%%%%%%%%%%%%%%%%%%%%%%%%%%%%%%%%%%%
% %% Version control history
% %%%%%%%%%%%%%%%%%%%%%%%%%%%%%%%%%%%%%%%%%%%%%%%%%%%%%%%%%%%%%%%
% %% $Id$
% %% $Log$
% %% Revision 1.3  2003/08/18 08:06:11  gawthrop
% %% A working version with more explantion
% %%
% %% Revision 1.2  2002/09/23 11:16:27  gawthrop
% %% New example for ident representation
% %%
% %% Revision 1.1  2001/04/05 11:57:29  gawthrop
% %% Identification example
% %%
% %% Revision 1.1  2000/12/28 09:13:38  peterg
% %% Initial revision
% %%
% %%%%%%%%%%%%%%%%%%%%%%%%%%%%%%%%%%%%%%%%%%%%%%%%%%%%%%%%%%%%%%%

   The acausal bond graph of system \textbf{idRC} is
   displayed in Figure \Ref{fig:idRC_abg.ps} and its label
   file is listed in Section \Ref{sec:idRC_lbl}.
   The subsystems are listed in Section \Ref{sec:idRC_sub}.

   
   This example illustrates the sensitivity approach to model-based
   system identification\footnote{Peter J Gawthrop, \emph{Sensitivity
       Bond Graphs}, Journal Franklin Institute, \textbf{337}, 2000,
     pp 907--922}.

   The system is a simple RC circuit with zero initial condition; the
   method identifies the resitance $r$.

   The data is created by typing:
\begin{verbatim}
make
\end{verbatim}
To see the results, type: 
\begin{verbatim}
mtt -oct -i euler  idRC ident view
\end{verbatim}

   \paragraph{NB} All sensitivity coefficients in idRC\_simpar.txt must
   be set to zero.