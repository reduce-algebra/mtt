% Verbal description for system TwoLinkxyc (TwoLinkxyc_desc.tex)
% Generated by MTT on Fri Jun 13 16:30:23 BST 1997.

% %%%%%%%%%%%%%%%%%%%%%%%%%%%%%%%%%%%%%%%%%%%%%%%%%%%%%%%%%%%%%%%
% %% Version control history
% %%%%%%%%%%%%%%%%%%%%%%%%%%%%%%%%%%%%%%%%%%%%%%%%%%%%%%%%%%%%%%%
% %% $Id$
% %% $Log$
% Revision 1.1  1997/08/15  13:31:00  peterg
% Initial revision
%
% %%%%%%%%%%%%%%%%%%%%%%%%%%%%%%%%%%%%%%%%%%%%%%%%%%%%%%%%%%%%%%%

   The acausal bond graph of system \textbf{TwoLinkxyc} is
   displayed in Figure \Ref{TwoLinkxyc_abg} and its label
   file is listed in Section \Ref{sec:TwoLinkxyc_lbl}.
   The subsystems are listed in Section \Ref{sec:TwoLinkxyc_sub}.

This system is identical to  \textbf{twolink} except that the two
colocated {\bf SS} components act at the tip in the $x$ and $y$
directions instead of at the two joints.

It uses two compound components: {\bf ROD} and {\bf GRAV}.  {\bf ROD}
is essentially as described in Figure 10.2 of "Metamodelling" and {\bf
GRAV} represents gravity by a vertical acceleration as in Section
10.9 of "Metamodelling"

%%% Local Variables: 
%%% mode: plain-tex
%%% TeX-master: t
%%% End: 
