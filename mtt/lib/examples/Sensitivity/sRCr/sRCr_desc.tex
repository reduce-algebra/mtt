% Verbal description for system sRCr (sRCr_desc.tex)
% Generated by MTT on Sun Aug 24 11:03:55 BST 1997.

% %%%%%%%%%%%%%%%%%%%%%%%%%%%%%%%%%%%%%%%%%%%%%%%%%%%%%%%%%%%%%%%
% %% Version control history
% %%%%%%%%%%%%%%%%%%%%%%%%%%%%%%%%%%%%%%%%%%%%%%%%%%%%%%%%%%%%%%%
% %% $Id$
% %% $Log$
% %% Revision 1.1  1999/07/29 05:17:04  peterg
% %% Initial revision
% %%
% %% Revision 1.1  1997/08/24 10:27:18  peterg
% %% Initial revision
% %%
% %%%%%%%%%%%%%%%%%%%%%%%%%%%%%%%%%%%%%%%%%%%%%%%%%%%%%%%%%%%%%%%

   The acausal bond graph of system \textbf{sRCr} is
   displayed in Figure \Ref{sRCr_abg} and its label
   file is listed in Section \Ref{sec:sRCr_lbl}.
   The subsystems are listed in Section \Ref{sec:sRCr_sub}.

The system \textbf{sRCr} is the the sensitivity version  of the simple
electrical sRCr circuit shown in Figure \Ref{sRCr_abg}. The circuit itself can be
regarded as a single-input single-output system with input $e_1$ and
output $e_2$; the sensitivity system has {\em two\/} outputs: $e_2$
and $\frac{\partial e_2}{\partial r}$.

All bonds are two-bond vector bonds, and the {\bf sR} and {\bf sC}
components are two-port versions of the usual {\bf R} and {\bf C}
components respectively. One port conveys the usual effort/flow pair;
the other port conveys the sensitivity of the effort and flow with
respect to the $r$ parameter.

