% Verbal description for system HeatedRod (HeatedRod_desc.tex)
% Generated by MTT on Thu Sep 4 18:05:09 BST 1997.

% %%%%%%%%%%%%%%%%%%%%%%%%%%%%%%%%%%%%%%%%%%%%%%%%%%%%%%%%%%%%%%%
% %% Version control history
% %%%%%%%%%%%%%%%%%%%%%%%%%%%%%%%%%%%%%%%%%%%%%%%%%%%%%%%%%%%%%%%
% %% $Id$
% %% $Log$
% %%%%%%%%%%%%%%%%%%%%%%%%%%%%%%%%%%%%%%%%%%%%%%%%%%%%%%%%%%%%%%%





   The acausal bond graph of system \textbf{HeatedRod} is
   displayed in Figure \Ref{HeatedRod_abg} and its label
   file is listed in Section \Ref{sec:HeatedRod_lbl}.
The subsystems are listed in Section \Ref{sec:HeatedRod_sub}.

 System \textbf{HeatedRod} is a model of a rod of copper with an
 electric current passing through it which warms it up. The two ends of
 the rod are fixed at ambient temperature; this is where all the heat
 loss occurs. 

 This system introduces the idea of the {\bf ES} component which
 transforms a relative-temperature/enthalpy pseudo bond (at the [e]
 port) into an absolute-temperature/enntropy energy bond (at the [s]
 port) and vice versa.

 The model is similar to that described in chapter 8 of
 Cellier's book. However, instead of representing the thermal
 resistance by {\bf RS} components and reinserting the entropy flow,
 the {\bf RT} component uses two {\bf ES} components to convert from
 true to pseudo bonds and back again. Similary, the thermal capacity is
 modelled by the {\bf CT} component.


 The rod parameters are given in the numpar file and the input current
 in the input file.
