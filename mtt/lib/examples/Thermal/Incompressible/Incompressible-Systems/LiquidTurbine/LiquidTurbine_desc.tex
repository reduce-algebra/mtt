% -*-latex-*- used to set EMACS into LaTeX-mode
% Verbal description for system LiquidTurbine (LiquidTurbine_desc.tex)
% Generated by MTT on Tue Jan 13 18:01:55 GMT 1998.

% %%%%%%%%%%%%%%%%%%%%%%%%%%%%%%%%%%%%%%%%%%%%%%%%%%%%%%%%%%%%%%%
% %% Version control history
% %%%%%%%%%%%%%%%%%%%%%%%%%%%%%%%%%%%%%%%%%%%%%%%%%%%%%%%%%%%%%%%
% %% $Id$
% %% $Log$
% %%%%%%%%%%%%%%%%%%%%%%%%%%%%%%%%%%%%%%%%%%%%%%%%%%%%%%%%%%%%%%%

   The acausal bond graph of system \textbf{LiquidTurbine} is
   displayed in Figure \Ref{LiquidTurbine_abg} and its label
   file is listed in Section \Ref{sec:LiquidTurbine_lbl}.
   The subsystems are listed in Section \Ref{sec:LiquidTurbine_sub}.

\textbf{LiquidTurbine} can be regarded as a single-spool gas turbine
with an incompressible working fluid. Of course, such a device cannot
convert heat to work; however, it provides a useful first step towards
modelling a gas turbine.

There are three main components:
\begin{enumerate}
\item p1 -- a leaky pump \textbf{lPump} component. This is analogous
  to the gas turbine compressor.
\item c1 -- a tank \textbf{Tank} component. This is analogous
  to the gas turbine combustion chamber.
\item t1 -- a leaky turbine \textbf{lTurb} component. This is analogous
  to the gas turbine turbine.
\end{enumerate}
The components \textbf{In} and \textbf{Out} provide the inlet and
outlet conditions.

%%% Local Variables: 
%%% mode: latex
%%% TeX-master: t
%%% End: 
