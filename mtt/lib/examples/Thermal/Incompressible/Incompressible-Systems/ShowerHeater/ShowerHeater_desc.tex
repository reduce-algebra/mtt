% -*-latex-*- used to set EMACS into LaTeX-mode
% Verbal description for system ShowerHeater (ShowerHeater_desc.tex)
% Generated by MTT on Tue Jan 13 18:01:55 GMT 1998.

% %%%%%%%%%%%%%%%%%%%%%%%%%%%%%%%%%%%%%%%%%%%%%%%%%%%%%%%%%%%%%%%
% %% Version control history
% %%%%%%%%%%%%%%%%%%%%%%%%%%%%%%%%%%%%%%%%%%%%%%%%%%%%%%%%%%%%%%%
% %% $Id$
% %% $Log$
% %%%%%%%%%%%%%%%%%%%%%%%%%%%%%%%%%%%%%%%%%%%%%%%%%%%%%%%%%%%%%%%

   The acausal bond graph of system \textbf{ShowerHeater} is
   displayed in Figure \Ref{ShowerHeater_abg} and its label
   file is listed in Section \Ref{sec:ShowerHeater_lbl}.
   The subsystems are listed in Section \Ref{sec:ShowerHeater_sub}.

\textbf{ShowerHeater} is a very elementary model of an electric heater suitable for
a shower. It illustates the use of bond graph components which are
internally pseudo, but externally true bond graphs
(temperature/entropy flow).

There are three main components:
\begin{enumerate}
\item p1 and p2 -- a \textbf{Pipe} component (see Section
  \Ref{sec:Pipe}). It is assumed that the pipes have zero flow
  resistance and thus do not generate heat via flow resistance.
\item t1 -- a tank \textbf{Tank} component. 
\item Heater -- a resistive heater modelled by the thermodynamic
  \textbf{R} component \textbf{RS}.
\end{enumerate}
Other components could be added to represent thermal conduction and
thermal capacities.

The components \textbf{In} and \textbf{Out} provide the inlet and
outlet conditions.

The three inputs are
\begin{description}
\item[$u_1$] The flow rate
\item[$u_2$] The inlet temperature
\item[$u_3$] The voltage across the heating element.
\end{description}
The single output is
\begin{description}
\item[$y_1$] The outflow temperature
\end{description}
and the state is 
\begin{description}
\item[$x_1$] The heat contained in the tank.
\end{description}

%%% Local Variables: 
%%% mode: latex
%%% TeX-master: t
%%% End: 
