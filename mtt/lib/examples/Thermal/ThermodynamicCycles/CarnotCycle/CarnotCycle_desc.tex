% Verbal description for system CarnotCycle (CarnotCycle_desc.tex)
% Generated by MTT on Tue Dec 9 12:13:57 GMT 1997.

% %%%%%%%%%%%%%%%%%%%%%%%%%%%%%%%%%%%%%%%%%%%%%%%%%%%%%%%%%%%%%%%
% %% Version control history
% %%%%%%%%%%%%%%%%%%%%%%%%%%%%%%%%%%%%%%%%%%%%%%%%%%%%%%%%%%%%%%%
% %% $Id$
% %% $Log$
% %%%%%%%%%%%%%%%%%%%%%%%%%%%%%%%%%%%%%%%%%%%%%%%%%%%%%%%%%%%%%%%

   The acausal bond graph of system \textbf{CarnotCycle} is
   displayed in Figure \Ref{CarnotCycle_abg} and its label
   file is listed in Section \Ref{sec:CarnotCycle_lbl}.
   The subsystems are listed in Section \Ref{sec:CarnotCycle_sub}.

The Carnot cycle is a simple closed thermodynamic cycle with four parts:
\begin{enumerate}
\item Isentropic compression
\item Heat injection at constant temperature
\item Isentropic expansion
\item Heat extraction at constant temperature
\end{enumerate}

The subsystem \textbf{Cycle} (Section \Ref{sec:Cycle}) is a two-port
component describing an ideal gas. It has two energy ports which, with
integral causality correspond to
\begin{enumerate}
\item Entropy flow in; temperature out
\item Volume rate of change in; pressure out
\end{enumerate}

In contast to the Otto cycle (see Table
\Ref{tab:cycles} where each table entry gives the causality on the
heat and work ports respectively). The ideal Carnot cycle has
derivative causality on the {\bf [Heat]} port for two parts of the
cycle.

To avoid this causlity change, the Carnot cycle is approximated by
applying the heat from a temperature source via a thermal resistance
{\bf RT} component. During the {\em heat injection\/} and {\em heat
extraction\/} parts of the cycle, the resistance parameter $r\approx
0$, but during the {\em isentropic compression\/} and {\em isentropic
expansion\/} parts of the cycle, the resistance parameter $r\approx
\inf$.

The simulation parameters appear in Section
\Ref{sec:CarnotCycle_numpar.txt}. The results are plotted against time
as follows:
\begin{itemize}
\item Volume (Figure \Ref{fig:CarnotCycle_odeso.ps-CarnotCycle-cycle-V})
\item Pressure (Figure
\Ref{fig:CarnotCycle_odeso.ps-CarnotCycle-cycle-P})
\item Entropy (Figure \Ref{fig:CarnotCycle_odeso.ps-CarnotCycle-cycle-S})
\item Temperature (Figure
\Ref{fig:CarnotCycle_odeso.ps-CarnotCycle-cycle-T})
\end{itemize}

These values are replotted as the standard PV and TS diagrams in
Figures
\Ref{fig:CarnotCycle_odeso.ps-CarnotCycle-cycle-V:CarnotCycle-cycle-P}
and
\Ref{fig:CarnotCycle_odeso.ps-CarnotCycle-cycle-S:CarnotCycle-cycle-T}
respectively.

The PV diagram shows the long and thin form typical of the Carnot
cycle -- this implies a poor work ratio. The TS diagram is not
informative; it is not the expected rectangle because both T and S
change in a stepwise manner.




