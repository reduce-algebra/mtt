% Verbal description for system INTF (INTF_desc.tex)
% Generated by MTT on Fri Aug 15 09:53:16 BST 1997.

% %%%%%%%%%%%%%%%%%%%%%%%%%%%%%%%%%%%%%%%%%%%%%%%%%%%%%%%%%%%%%%%
% %% Version control history
% %%%%%%%%%%%%%%%%%%%%%%%%%%%%%%%%%%%%%%%%%%%%%%%%%%%%%%%%%%%%%%%
% %% $Id$
% %% $Log$
% %% Revision 1.1  1997/08/24 11:20:18  peterg
% %% Initial revision
% %%
% %%%%%%%%%%%%%%%%%%%%%%%%%%%%%%%%%%%%%%%%%%%%%%%%%%%%%%%%%%%%%%%

   The acausal bond graph of system \textbf{INTFS} is
   displayed in Figure \Ref{INTFS_abg} and its label
   file is listed in Section \Ref{sec:INTFS_lbl}.
   The subsystems are listed in Section \Ref{sec:INTFS_sub}.

\textbf{INTFS} is a two-port component where the effort on port [out]
   is the integral of the flow on port [in]. The single parameter e_0
   is the initial value of the integral.
