% -*-latex-*- Put EMACS into LaTeX-mode
% Verbal description for system Pump (Pump_desc.tex)
% Generated by MTT on Fri Mar 20 15:53:12 GMT 1998.

% %%%%%%%%%%%%%%%%%%%%%%%%%%%%%%%%%%%%%%%%%%%%%%%%%%%%%%%%%%%%%%%
% %% Version control history
% %%%%%%%%%%%%%%%%%%%%%%%%%%%%%%%%%%%%%%%%%%%%%%%%%%%%%%%%%%%%%%%
% %% $Id$
% %% $Log$
% %%%%%%%%%%%%%%%%%%%%%%%%%%%%%%%%%%%%%%%%%%%%%%%%%%%%%%%%%%%%%%%

   The acausal bond graph of system \textbf{Pump} is
   displayed in Figure \Ref{Pump_abg} and its label
   file is listed in Section \Ref{sec:Pump_lbl}.
   The subsystems are listed in Section \Ref{sec:Pump_sub}.

\textbf{Pump} represents an ideal pumping component for compressible
or incompressible flow though a pipe. 

The pump is ideal in the sense that the mass flow rate $\dot m$
depends only on the shaft speed $\omega$:
\begin{equation}
  \dot m = k_p \omega
\end{equation}

It is implemented using three components:
\begin{itemize}
\item the ideal isentropic \textbf{wPipe} component which gives the
  correct energy flows
\item the polytropic expansion \textbf{Poly} component which imposes
  the correct temperature at the output of the pump. This component
  imposes zero flow at all its ports and therefore does not affect
  energy balance. It has a bicausal output imposing both the
  temperature measured by the \textbf{SS} component ``T'' and a zero
  flow.
\item the \emph{effort-bicausal transformer} \textbf{EBTF}
  component. This component is an energy-conserving \textbf{TF}
  component with non-standard causality. The modulus is determined by
  the two imposed efforts ($T$ and $T_2$), and this modulus determines
  the flows in the usual way. In particular, it makes sure that the
  internal energy flowing from the pump to the following components
  (imposing $T_2$) is correct.
\end{itemize}
