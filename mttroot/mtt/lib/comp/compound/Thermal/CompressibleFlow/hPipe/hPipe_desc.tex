% -*-latex-*- Put EMACS into LaTeX-mode
% Verbal description for system hPipe (hPipe_desc.tex)
% Generated by MTT on Tue Mar 31 09:54:08 BST 1998.

% %%%%%%%%%%%%%%%%%%%%%%%%%%%%%%%%%%%%%%%%%%%%%%%%%%%%%%%%%%%%%%%
% %% Version control history
% %%%%%%%%%%%%%%%%%%%%%%%%%%%%%%%%%%%%%%%%%%%%%%%%%%%%%%%%%%%%%%%
% %% $Id$
% %% $Log$
% %%%%%%%%%%%%%%%%%%%%%%%%%%%%%%%%%%%%%%%%%%%%%%%%%%%%%%%%%%%%%%%

   The acausal bond graph of system \textbf{hPipe} is
   displayed in Figure \Ref{hPipe_abg} and its label
   file is listed in Section \Ref{sec:hPipe_lbl}.
   The subsystems are listed in Section \Ref{sec:hPipe_sub}.
   

\textbf{hPipe} represents an ideal (energy conserving) pipe carrying a
fluid with heat transfer. To ensure energy conservation, power bonds are used and
connected by (energy conserving) \textbf{TF} components.
It is assumed that the working fluid is an ideal gas (gas constant $r$) and that a mass
$m_t$ is stored within pipe with a volume $v_t$.

The central \textbf{0} junction carries temperature ($T$) and the two
hydraulic ports are connected to this by appropriate transformers.
The modulus of the \textbf{TF} component labeled ``P2T'' is such P and
T are related by the ideal gas law
\begin{equation}
  P = \frac{Rm_t}{v_t} T
\end{equation}

