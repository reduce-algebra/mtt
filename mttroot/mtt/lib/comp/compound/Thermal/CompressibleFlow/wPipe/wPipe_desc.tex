% Verbal description for system wPipe (wPipe_desc.tex)
% Generated by MTT on Mon Mar 9 09:11:22 GMT 1998.

% %%%%%%%%%%%%%%%%%%%%%%%%%%%%%%%%%%%%%%%%%%%%%%%%%%%%%%%%%%%%%%%
% %% Version control history
% %%%%%%%%%%%%%%%%%%%%%%%%%%%%%%%%%%%%%%%%%%%%%%%%%%%%%%%%%%%%%%%
% %% $Id$
% %% $Log$
% %%%%%%%%%%%%%%%%%%%%%%%%%%%%%%%%%%%%%%%%%%%%%%%%%%%%%%%%%%%%%%%

   The acausal bond graph of system \textbf{wPipe} is
   displayed in Figure \Ref{wPipe_abg} and its label
   file is listed in Section \Ref{sec:wPipe_lbl}.
   The subsystems are listed in Section \Ref{sec:wPipe_sub}.

\textbf{wPipe} represents an ideal (energy conserving) pipe carrying a
fluid with work transfer. To ensure energy conservation, power bonds are used and
connected by (energy conserving) \textbf{TF} components.

The central \textbf{1} junction carries mass flow ($\dot m$) and the
four ports are connected to this by appropriate transformers. In the
case of the hydraulic ports, these transformers are \emph{modulated}
by the corresponding fluid density.
The bonds impinging on this  \textbf{1} junction carry the
corresponding effort variables; in particular, the thermal bonds carry
specific internal energy $u$ and the hydraulic bonds carry $Pv$ where
$P$ is the pressure and $v$ the specific volume.

The ports ``Work\_in'' and ``Work\_out'' are convenient for attaching
(for example) the shadt work of a pump, turbine or compressor.

The ports are 
\begin{itemize}
\item [Hy\_in] Pressure/volume-flow inflow
\item [Hy\_in] Pressure/volume-flow outflow
\item [Th\_in] Temperature/Entropy-flow in flow
\item [Th\_out] Temperature/Entropy-flow out flow
\item [Shaft] Torque/angular velocity input.
\end{itemize}

%%% Local Variables: 
%%% mode: latex
%%% TeX-master: t
%%% End: 
