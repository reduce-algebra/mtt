% Verbal description for system Pipe (Pipe_desc.tex)
% Generated by MTT on Tue Jan 13 18:02:53 GMT 1998.

% %%%%%%%%%%%%%%%%%%%%%%%%%%%%%%%%%%%%%%%%%%%%%%%%%%%%%%%%%%%%%%%
% %% Version control history
% %%%%%%%%%%%%%%%%%%%%%%%%%%%%%%%%%%%%%%%%%%%%%%%%%%%%%%%%%%%%%%%
% %% $Id$
% %% $Log$
% %%%%%%%%%%%%%%%%%%%%%%%%%%%%%%%%%%%%%%%%%%%%%%%%%%%%%%%%%%%%%%%

   The acausal bond graph of system \textbf{Pipe} is
   displayed in Figure \Ref{Pipe_abg} and its label
   file is listed in Section \Ref{sec:Pipe_lbl}.
   The subsystems are listed in Section \Ref{sec:Pipe_sub}.

The \textbf{Pipe} component represents one way flow of incompressible
fluid though a pipe. Externally, it has true energy bonds: $P$/$\dot V$
(Pressure/volume-flow) representing hydraulic energy and $T$/$\dot
S$(Temperature/Entropy-flow) representing convected thermal energy.

Internally, however, the thermal part is represented by a pseudo bond
graph which computes the flow of internal energy $\dot E$ from the
upstream temperature $T_1$ and the volumetric flow rate $\dot V$ as:
\begin{equation}
  \dot E = \rho c_p T_1 \dot V
\end{equation}
The $AF$ component makes the $FMR$ component use $T_1$ rather than
$T_1-T_2$.

The two \textbf{ES} components provide the conversion from true to
psuedo thermal bonds and vice versa.


%%% Local Variables: 
%%% mode: latex
%%% TeX-master: t
%%% End: 
