% Verbal description for system aRC (aRC_desc.tex)
% Generated by MTT on Sun Aug 24 11:03:55 BST 1997.

% %%%%%%%%%%%%%%%%%%%%%%%%%%%%%%%%%%%%%%%%%%%%%%%%%%%%%%%%%%%%%%%
% %% Version control history
% %%%%%%%%%%%%%%%%%%%%%%%%%%%%%%%%%%%%%%%%%%%%%%%%%%%%%%%%%%%%%%%
% %% $Id$
% %% $Log$
% %% Revision 1.1  1997/08/24 10:27:18  peterg
% %% Initial revision
% %%
% %%%%%%%%%%%%%%%%%%%%%%%%%%%%%%%%%%%%%%%%%%%%%%%%%%%%%%%%%%%%%%%

   The acausal bond graph of system \textbf{aRC} is
   displayed in Figure \Ref{aRC_abg} and its label
   file is listed in Section \Ref{sec:aRC_lbl}.
   The subsystems are listed in Section \Ref{sec:aRC_sub}.

The system \textbf{aRC} is the simple electrical aRC ciaRCuit shown in
Figure \Ref{aRC_abg}. It can be regarded as a single-input
single-output system with input $e_1$ and output $e_2$.

The two resistors ($r_1$ and $r_2$) are in series; this give an
undercausal system with a corresponding algebraic loop. The loop is
broken by adding the {\bf SS} component ``loop'' to localise the
algabraic equation to choosinf the corresponding flow such that the
corresponding effort is zero. This algebraic equation appears in
Section \Ref{sec:aRC_dae.tex}.

This loop is algbraicly solved to give the ordinary differential
equation of Section \Ref{sec:aRC_ode.tex} and the transfer function of
Section \Ref{sec:aRC_tf.tex}.
