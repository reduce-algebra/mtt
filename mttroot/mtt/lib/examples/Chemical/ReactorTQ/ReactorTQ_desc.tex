% -*-latex-*- Put EMACS into LaTeX-mode
% Verbal description for system ReactorTQ (ReactorTQ_desc.tex)
% Generated by MTT on Fri Mar 3 12:43:33 GMT 2000.

% %%%%%%%%%%%%%%%%%%%%%%%%%%%%%%%%%%%%%%%%%%%%%%%%%%%%%%%%%%%%%%%
% %% Version control history
% %%%%%%%%%%%%%%%%%%%%%%%%%%%%%%%%%%%%%%%%%%%%%%%%%%%%%%%%%%%%%%%
% %% $Id$
% %% $Log$
% %%%%%%%%%%%%%%%%%%%%%%%%%%%%%%%%%%%%%%%%%%%%%%%%%%%%%%%%%%%%%%%

\fig{ReactorTQ_pic}
{ReactorTQ_pic} {0.9} {System \textbf{ReactorTQ}, Schematic}

Figure \Ref{fig:ReactorTQ_pic} is the schematic diagram of a chemical
reactor.

The acausal bond graph of system \textbf{ReactorTQ} is displayed in
Figure \Ref{fig:ReactorTQ_abg.ps} and its label file is listed in
Section \Ref{sec:ReactorTQ_lbl}.  The subsystems are listed in Section
\Ref{sec:ReactorTQ_sub}.

This example of a (nonlinear) chemical reactor is due to Trickett and
Bogle\footnote{ K. J. Tricket, \emph{Quantification of Inverse
    Responses for Controllability Assessment of Nonlinear Processes},
  PhD Thesis, University College London, 1994} is used in this
section.  The reactor has two reaction mechanisms: $\text{A}
\rightarrow \text{B} \rightarrow \text{C}$ and $\text{2A} \rightarrow
\text{D}$.  The reactor mass inflow and outflow $f_r$ are identical.
$q$ represents the heat inflow to the reactor.

The control loop $t$/$q$ has been inverted. The resulting SISO
system has two interpretations:
\begin{enumerate}
\item the \emph{dynamics} of the $c_b$/$f$ loop when the $t$/$q$ loop
  is under perfect control and
\item the \emph{inverse} dynamics of the  $t$/$q$ loop.
\end{enumerate}

\fig{ReactorTQ_zero} {ReactorTQ_zero} {0.9}
{System\textbf{ReactorTQ}: zeros v flow} 

Figure \Ref{fig:ReactorTQ_zero}
shows the poles of the linearised system as the steady-state flow
varies: these are the \emph{zeros} of the $c_b$/$f$ control-loop when
the $t$/$q$ loop is \emph{open}.


