% Verbal description for system MacroMicroPPP (MacroMicroPPP_desc.tex)
% Generated by MTT on Tue Dec 9 17:55:41 GMT 1997.

% %%%%%%%%%%%%%%%%%%%%%%%%%%%%%%%%%%%%%%%%%%%%%%%%%%%%%%%%%%%%%%%
% %% Version control history
% %%%%%%%%%%%%%%%%%%%%%%%%%%%%%%%%%%%%%%%%%%%%%%%%%%%%%%%%%%%%%%%
% %% $Id$
% %% $Log$
% %% Revision 1.1  2000/05/21 16:10:07  peterg
% %% Initial revision
% %%
% %%%%%%%%%%%%%%%%%%%%%%%%%%%%%%%%%%%%%%%%%%%%%%%%%%%%%%%%%%%%%%%

   The acausal bond graph of system \textbf{MacroMicroPPP} is
   displayed in Figure \Ref{MacroMicroPPP_abg} and its label
   file is listed in Section \Ref{sec:MacroMicroPPP_lbl}.
   The subsystems are listed in Section \Ref{sec:MacroMicroPPP_sub}.

This is a Bond Graph model of the macro-micro manipulation system
discussed by Sharon in his thesis and BY Sharon, Hogan and Hardt in
various papers.

It can be regarded as a single-input single-output system with input
$e_1$ and output $e_2$.  The bond graph in Figure \Ref{rcPPP_abg} is
augmented with the {\em open-loop\/} controller comprising
\begin{itemize}
\item seven effort source \textbf{Se} components ``u1'' to``u7''
\item seven effort amplifier \textbf{AE} components ``ppp\_1'' to `ppp\_7''
\end{itemize}
This allows up to 7 input basis functions.

This is a \emph{linear} system; but it is used to illustate
\emph{nonlinear} PPP control. 
Section \Ref{sec:rcPPP_nppp.ps} compares
\begin{itemize}
\item linear open-loop PPP control
\item linear closed-loop PPP control
\item nonlinear PPP control
\end{itemize}
as linear open and closed-loop control.
