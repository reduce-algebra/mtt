% Verbal description for system rcPPP (rcPPP_desc.tex)
% Generated by MTT on Sun Aug 24 11:03:55 BST 1997.

% %%%%%%%%%%%%%%%%%%%%%%%%%%%%%%%%%%%%%%%%%%%%%%%%%%%%%%%%%%%%%%%
% %% Version control history
% %%%%%%%%%%%%%%%%%%%%%%%%%%%%%%%%%%%%%%%%%%%%%%%%%%%%%%%%%%%%%%%
% %% $Id$
% %% $Log$
% %% Revision 1.1  1997/08/24 10:27:18  peterg
% %% Initial revision
% %%
% %%%%%%%%%%%%%%%%%%%%%%%%%%%%%%%%%%%%%%%%%%%%%%%%%%%%%%%%%%%%%%%

The acausal bond graph of system \textbf{rcPPP} is
displayed in Figure \Ref{rcPPP_abg} and its label
file is listed in Section \Ref{sec:rcPPP_lbl}.
The subsystems are listed in Section \Ref{sec:rcPPP_sub}.
The system \textbf{rcPPP} is based on the simple electrical rc circuit shown in
Figure \Ref{rcPPP_abg}. 

It can be regarded as a single-input
single-output system with input $e_1$ and output $e_2$.
The bond graph in Figure \Ref{rcPPP_abg} is augmented with the {\em
open-loop\/} controller comprising
\begin{itemize}
\item two effort source \textbf{Se} components ``u1'' and ``u2''
\item two effort amplifier \textbf{AE} components ``ppp\_1'' and `ppp\_2''
\end{itemize}

This is a \emph{linear} system; but it is used to illustate
\emph{nonlinear} PPP control. Not surprisingly, the \emph{nonlinear}
PPP control gives nearly the same result (Section \Ref{sec:rcPPP_nppp.ps})
as linear open and closed-loop control.





