% -*-latex-*- Put EMACS into LaTeX-mode
% Verbal description for system Weir (Weir_desc.tex)
% Generated by MTT on Tue Mar 2 22:05:29 GMT 1999.

% %%%%%%%%%%%%%%%%%%%%%%%%%%%%%%%%%%%%%%%%%%%%%%%%%%%%%%%%%%%%%%%
% %% Version control history
% %%%%%%%%%%%%%%%%%%%%%%%%%%%%%%%%%%%%%%%%%%%%%%%%%%%%%%%%%%%%%%%
% %% $Id$
% %% $Log$
% %%%%%%%%%%%%%%%%%%%%%%%%%%%%%%%%%%%%%%%%%%%%%%%%%%%%%%%%%%%%%%%

   The acausal bond graph of system \textbf{Weir} is
   displayed in Figure \Ref{Weir_abg} and its label
   file is listed in Section \Ref{sec:Weir_lbl}.
   The subsystems are listed in Section \Ref{sec:Weir_sub}.


The weir is modelled by an \textbf{ISW} component in series with an
\textbf{R} component. Physicaly, the former represents the inertia of
the fluid together with the switching effect of the weir; the latter
represents the flow resistance.

The switching logic is on if the level on either side of the weir
reaches the level of the weir.
