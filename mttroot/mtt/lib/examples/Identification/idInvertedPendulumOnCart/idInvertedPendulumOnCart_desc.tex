% -*-latex-*- Put EMACS into LaTeX-mode
% Verbal description for system idInvertedPendulumOnCart (InvertedPendulumOnCart_desc.tex)
% Generated by MTT on Thu Aug 5 10:38:16 BST 1999.

% %%%%%%%%%%%%%%%%%%%%%%%%%%%%%%%%%%%%%%%%%%%%%%%%%%%%%%%%%%%%%%%
% %% Version control history
% %%%%%%%%%%%%%%%%%%%%%%%%%%%%%%%%%%%%%%%%%%%%%%%%%%%%%%%%%%%%%%%
% %% $Id$
% %% $Log$
% %% Revision 1.1  2000/12/28 18:00:45  peterg
% %% To RCS
% %%
% %%%%%%%%%%%%%%%%%%%%%%%%%%%%%%%%%%%%%%%%%%%%%%%%%%%%%%%%%%%%%%%

   The acausal bond graph of system \textbf{idInvertedPendulumOnCart} is
   displayed in Figure \Ref{fig:InvertedPendulumOnCart_abg.ps} and its label
   file is listed in Section \Ref{sec:InvertedPendulumOnCart_lbl}.
   The subsystems are listed in Section \Ref{sec:InvertedPendulumOnCart_sub}.

This is a one input, two output nonlinear system comprising an
inverted pendulum attached by a hinge to a cart constrained to move in
the horizontal direction. The input is the horizontal force acting on
the cart, and the two outputs are the horizontal position and the
pendulum angle respectively.

The identification procedure estimates the two friction parameters
$r_c$ and $r_c$