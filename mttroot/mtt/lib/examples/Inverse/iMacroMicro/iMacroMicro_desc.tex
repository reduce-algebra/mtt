% Verbal description for system iMacroMicro (iMacroMicro_desc.tex)
% Generated by MTT on Tue Dec 9 18:13:04 GMT 1997.

% %%%%%%%%%%%%%%%%%%%%%%%%%%%%%%%%%%%%%%%%%%%%%%%%%%%%%%%%%%%%%%%
% %% Version control history
% %%%%%%%%%%%%%%%%%%%%%%%%%%%%%%%%%%%%%%%%%%%%%%%%%%%%%%%%%%%%%%%
% %% $Id$
% %% $Log$
% %% Revision 1.1  1999/02/22 22:24:27  peterg
% %% Initial revision
% %%
% %%%%%%%%%%%%%%%%%%%%%%%%%%%%%%%%%%%%%%%%%%%%%%%%%%%%%%%%%%%%%%%

   The acausal bond graph of system \textbf{iMacroMicro} is
   displayed in Figure \Ref{iMacroMicro_abg} and its label
   file is listed in Section \Ref{sec:iMacroMicro_lbl}.
   The subsystems are listed in Section \Ref{sec:iMacroMicro_sub}.

This is a Bond Graph model of the macro-micro manipulation system
discussed by Sharon in his thesis and by Sharon, Hogan and Hardt in
various papers. The micro loop is inverted whilst leaving the macro
control in place. This {\em partial inverse\/} gives information about
the {\em zero dynamics\/} of the micro control system with the
particular macro controller in place and allows desidn of the macro
controller to ease the design of the micro controller.


%%% Local Variables: 
%%% mode: plain-tex
%%% TeX-master: t
%%% End: 
