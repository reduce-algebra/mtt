% -*-latex-*- Put EMACS into LaTeX-mode
% Verbal description for system iTanks (iTanks_desc.tex)
% Generated by MTT on Wed Nov 18 11:04:33 GMT 1998.

% %%%%%%%%%%%%%%%%%%%%%%%%%%%%%%%%%%%%%%%%%%%%%%%%%%%%%%%%%%%%%%%
% %% Version control history
% %%%%%%%%%%%%%%%%%%%%%%%%%%%%%%%%%%%%%%%%%%%%%%%%%%%%%%%%%%%%%%%
% %% $Id$
% %% $Log$
% %%%%%%%%%%%%%%%%%%%%%%%%%%%%%%%%%%%%%%%%%%%%%%%%%%%%%%%%%%%%%%%

Figure \Ref{iTanks_abg} shows the bond graph of a two-tank system
superimposed on a schematic diagram. 
The two \textbf{C} components corresponds to the fluid storage and how
it relates to the pressure at the base of the tanks. In this case, for
simplicity, each tank ($i=1$ or $i=2$) is assumed to have a unity constitutive relationship:
\begin{equation}
  \text{pressure} = p_i = v_i = \text{volume}
\end{equation}
The volumetric flow rate into the first, and out of the second, tank
is represented by the two unlabelled \textbf{R} components. Again,
each is assumed to have a unit constitutive relationship:
\begin{equation}
  \text{flow} = f_i = \Delta_i = \text{pressure drop}
\end{equation}
The volumetric flow rate between the first and the second  tanks
is represented  \textbf{R} component labelled $k$. The constitutive relationship is assumed
linear of the form:
\begin{equation}
  \text{flow} = f = k \Delta  = \text{pressure drop}
\end{equation}

The system has two inputs:
\begin{equation}
  \begin{aligned}
    u_1 &= \text{input pressure at left-hand pipe} \\
    u_2 &= \text{input pressure at right-hand pipe} 
  \end{aligned}
\end{equation}
and two outputs:
\begin{equation}
  \begin{aligned}
    y_1 &= p_1 = \text{pressure at left-hand tank} \\
    y_2 &= p_2 = \text{pressure at right-hand tank} 
  \end{aligned}
\end{equation}
The system transfer-function matrix is given by:
\begin{equation}
  \begin{aligned}
G_{11} = G_{22} &= \frac{(s + k + 1)}{(s^2 + 2 s {(k + 1)} + 2 k + 1)}\\
G_{12} = G_{21} &= \frac{k}{(s^2 + 2 s {(k + 1)} + 2 k + 1)}
\end{aligned}
\end{equation}

However, Figure \Ref{iTanks_abg} shows the causality of the
\textbf{SS} components to \emph{invert} the system with respect to its
inputs and outputs. Figure \Ref{fig:iTanks_cbg.ps} shows the causally
complete bond graph; this system has no dynamic components in integral
causality -- the inverse has no poles and therefore the system has no
zeros.

Some further representations of the inverse appear in the following
sections.


