% -*-latex-*- Put EMACS into LaTeX-mode
% Verbal description for system PinnedBeam (PinnedBeam_desc.tex)
% Generated by MTT on Mon Apr 19 07:04:54 BST 1999.

% %%%%%%%%%%%%%%%%%%%%%%%%%%%%%%%%%%%%%%%%%%%%%%%%%%%%%%%%%%%%%%%
% %% Version control history
% %%%%%%%%%%%%%%%%%%%%%%%%%%%%%%%%%%%%%%%%%%%%%%%%%%%%%%%%%%%%%%%
% %% $Id$
% %% $Log$
% %% Revision 1.1  1999/05/18 04:01:50  peterg
% %% Initial revision
% %%
% %%%%%%%%%%%%%%%%%%%%%%%%%%%%%%%%%%%%%%%%%%%%%%%%%%%%%%%%%%%%%%%

The acausal bond graph of system \textbf{PinnedBeam} is displayed in
Figure \Ref{fig:PinnedBeam_abg.ps} and its label file is listed in
Section \Ref{sec:PinnedBeam_lbl}.  The subsystems are listed in Section
\Ref{sec:PinnedBeam_sub}.
   
This example represents the dynamics of a uniform beam with two pinned
ends. The left-hand end is driven by a torque input and the
corresponding collocated angular velocity is measured.  The beam is
approximated by 20 equal lumps using the Bernoulli-Euler approximation
with damping. 

Because the two end lumps have different causality to the rest of the
beam lumps, they are represented seperately.

The system parameters are given in Section
\Ref{sec:PinnedBeam_numpar.tex}. 

 The system has 20 states (10
modes of vibration), 1 inputs and 1 outputs.

The first 5 vibration frequencies are given in Table \ref{tab:freq}
togtherr with the theoretical (based on the Bernoulli-Euler beam with
the same values of $EI$ and $\rho A$. 
\begin{table}[htbp]
  \begin{center}
    \begin{tabular}{||l|l|l||}
      \hline
      \hline
      Mode & Frequency & Theoretical frequency\\
      \hline
      1 & 119.44 & 119.69\\
      2 & 474.83 & 479.02\\
      3 &1057.41 &1078.09\\
      4 &1852.85 &1914.86\\
      5 &2841.54 &2992.95\\
      \hline
      \hline
    \end{tabular}
    \caption{Mode frequencies (rad $s^{-1}$)}
    \label{tab:freq}
  \end{center}
\end{table}





