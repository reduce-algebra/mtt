% Verbal description for system Pendulum (Pendulum_desc.tex)
% Generated by MTT on Fri Aug 15 09:52:55 BST 1997.

% %%%%%%%%%%%%%%%%%%%%%%%%%%%%%%%%%%%%%%%%%%%%%%%%%%%%%%%%%%%%%%%
% %% Version control history
% %%%%%%%%%%%%%%%%%%%%%%%%%%%%%%%%%%%%%%%%%%%%%%%%%%%%%%%%%%%%%%%
% %% $Id$
% %% $Log$
% %%%%%%%%%%%%%%%%%%%%%%%%%%%%%%%%%%%%%%%%%%%%%%%%%%%%%%%%%%%%%%%

   The acausal bond graph of system \textbf{Pendulum} is
   displayed in Figure \Ref{Pendulum_abg} and its label
   file is listed in Section \Ref{sec:Pendulum_lbl}.
   The subsystems are listed in Section \Ref{sec:Pendulum_sub}.

This is a heirachical version of the example from Section 10.3 of
``Metamodelling''.  It uses two compound components: {\bf ROD} and {\bf
GRAV}. {\bf ROD} is
essentially as described in Figure 10.2 {\bf GRAV} represents gravity by a
vertical accelleration as in Section 10.9 of "Metamodelling".

%%% Local Variables: 
%%% mode: plain-tex
%%% TeX-master: t
%%% End: 
