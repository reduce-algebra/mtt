% Verbal description for system gTwoLink (gTwoLink_desc.tex)
% Generated by MTT on Fri Jun 13 16:30:23 BST 1997.

% %%%%%%%%%%%%%%%%%%%%%%%%%%%%%%%%%%%%%%%%%%%%%%%%%%%%%%%%%%%%%%%
% %% Version control history
% %%%%%%%%%%%%%%%%%%%%%%%%%%%%%%%%%%%%%%%%%%%%%%%%%%%%%%%%%%%%%%%
% %% $Id$
% %% $Log$
% %% Revision 1.1  1998/01/19 14:20:07  peterg
% %% Initial revision
% %%
% Revision 1.1  1997/08/15  13:31:00  peterg
% Initial revision
%
% %%%%%%%%%%%%%%%%%%%%%%%%%%%%%%%%%%%%%%%%%%%%%%%%%%%%%%%%%%%%%%%

   The acausal bond graph of system \textbf{gTwoLink} is
   displayed in Figure \Ref{gTwoLink_abg} and its label
   file is listed in Section \Ref{sec:gTwoLink_lbl}.
   The subsystems are listed in Section \Ref{sec:gTwoLink_sub}.

This is a heirachical version of the example from Section 10.5 of
"Metamodelling".  It uses the compound components: {\bf ROD}.  {\bf
ROD} is essentially as described in Figure 10.2.
Gravity is included as discussed in "Metamodelling" by accelerating
the manipulator vertically using the {\bf ACCEL} component.

This system has a number of dynamic elements (those corresponding to
translation motion) in derivative causality, thus the system is
represnted as a Differential-Algebraic Equation (Section
\Ref{sec:gTwoLink_dae.tex}). Hovever, this
is of contrained-state form and therfore can be written as a set of
constrained-state equations (Section \Ref{sec:gTwoLink_cse.tex}). The
corresponding ordinary differential equation is complicated due to the
trig functions involved in inverting the E matrix.

As well as the standard representation the ``robot-form'' equations
appear in Section  \Ref{sec:gTwoLink_rfe.tex}. 

%%% Local Variables: 
%%% mode: plain-tex
%%% TeX-master: t
%%% End: 
