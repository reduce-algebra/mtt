% -*-latex-*- used to set EMACS into LaTeX-mode
% Verbal description for system SimpleGasTurbine (SimpleGasTurbine_desc.tex)
% Generated by MTT on Tue Jan 13 18:01:55 GMT 1998.

% %%%%%%%%%%%%%%%%%%%%%%%%%%%%%%%%%%%%%%%%%%%%%%%%%%%%%%%%%%%%%%%
% %% Version control history
% %%%%%%%%%%%%%%%%%%%%%%%%%%%%%%%%%%%%%%%%%%%%%%%%%%%%%%%%%%%%%%%
% %% $Id$
% %% $Log$
% %% Revision 1.1  1998/05/18 15:45:50  peterg
% %% Initial revision
% %%
% %%%%%%%%%%%%%%%%%%%%%%%%%%%%%%%%%%%%%%%%%%%%%%%%%%%%%%%%%%%%%%%

   The acausal bond graph of system \textbf{SimpleGasTurbine} is
   displayed in Figure \Ref{SimpleGasTurbine_abg} and its label
   file is listed in Section \Ref{sec:SimpleGasTurbine_lbl}.
   The subsystems are listed in Section \Ref{sec:SimpleGasTurbine_sub}.
   
   \textbf{SimpleGasTurbine} can be regarded as an single-spool gas
   turbine (producing shaft power) with an ideal-gas working fluid. It
   corresponds to the simple Joule Cycle as described in Chapter 12 of
   Rogers and Mayhew and in Chapter 2 of Cohen, Rogers and
   Saravanamutto. However, unlike those examples, the system is
   written with dynamics in mind.
   
   The system is described using an energy Bond Graph- this ensures
   that the first law is observed. In particular transformers are used
   to explicitly convert between energy covariables. Although this is
   a simple model, I believe that it provides the basis for building
   complex thermodynamic systems involving gas power cycles.


There are five main components:
\begin{enumerate}
\item p1 -- a \textbf{Pump} component representing the compressor
  stage. This converts shaft work to energy flow in the working fluid.
\item c1 -- a \textbf{Comb} component representing the combustion
  chamber. This converts the heat obtained by burning fuel to energy
  flow in the working fluid.
\item t1 -- a \textbf{Turb} component representing the turbine
  component. This converts the energy flow in the working fluid to
  shaft work
\item j\_s -- an \textbf{I} component representing the combined inertia
  of the shaft and compressor and turbine rotors.
\item a \textbf{Load} component to absorb the shaft power.
\end{enumerate}
The components \textbf{In} and \textbf{Out} provide the inlet and
outlet conditions.

Both \textbf{Pump} and \textbf{Turb} are implemented with the
\emph{polytropic} constitutive relationship with index $n$. When
$n=\gamma=\frac{c_p}{c_v}$ this corresponds to isentropic compression
and expansion and thus the \textbf{SimpleGasTurbine} achieves its
cycle efficiency. However, other values of $n$ can be used to account
for isentropic efficiency of less than unity.

To obtain a very simple dynamic model (and to avoid the need for an
accurate combustion chamber model) there are no dynamics associated
with the combustion chamber, but rahter it is assumed that the
corresponding temperature is imposed on the component (that is $T_3$
is the system input) the corresponding heat flow is then an output.

Both heat input and work output are measured using the \textbf{PS}
(power sensor) component, that for work output is embedded in the
\textbf{Load} component. These can be monitored to give the efficiency
of the \textbf{SimpleGasTurbine}.

A symbolic steady-state for the model was computed -- see Section
\ref{sec:SimpleGasTurbine_ss.tex}. In particular, the load
resistance was chosen to absorb all the generated work at the steady
state and the shaft inertia was chosen to give a unit time constant
for the linearised system. The mass flow and shaft speeds were taken
as unity.

For the purposed of simulation, the numerical values given in Examples
12.1 of Chapter 12 of Rogers and Mayhew, except that the isentropic
efficiencies are 100\% ($n=\gamma$) -- see Section
\ref{sec:SimpleGasTurbine_numpar.tex}.

Simulations were performed starting at the steady state and increasing
the combustion chamber temperature by 10\% at $t=1$ and reducing by
10\% at $t=5$. Graphs of the various outputs are plotted:
\begin{itemize}
\item Figure
  \Ref{fig:SimpleGasTurbine_odeso-SimpleGasTurbine-comp-1-T,SimpleGasTurbine-c1-1-T,SimpleGasTurbine-turb-1-T.ps}
  -- the temperatures at the output of the
  \begin{itemize}
  \item compressor,
  \item combustion chamber and
  \item turbine
  \end{itemize}
\item Figure
  \Ref{fig:SimpleGasTurbine_odeso-SimpleGasTurbine-fuel-1-Heat-1-y,SimpleGasTurbine-load-1-Work-1-y.ps}
  -- the heat input and work output
\item Figure
  \Ref{fig:SimpleGasTurbine_odeso-SimpleGasTurbine-shaft-1-speed-1-y.ps} -- the shaft speed and
\item Figure
  \Ref{fig:SimpleGasTurbine_odeso-SimpleGasTurbine-c1-1-P.ps}
  -- the pressure at the output of the
  \begin{itemize}
  \item compressor,
  \item combustion chamber and
  \item turbine
  \end{itemize}
\end{itemize}

This model can be modified extended in various ways to yield related
dynamic systems. For example:
\begin{itemize}
\item an air cooler is obtained by changing the direction of heat and
  work flows
\item additional \textbf{Turb} and \textbf{Comb} components add reheat
  to the cycle
\item an isentropic nozzle can be added and the work output removed
  to give a jet engine.
\end{itemize}


%%% Local Variables: 
%%% mode: latex
%%% TeX-master: t
%%% End: 
