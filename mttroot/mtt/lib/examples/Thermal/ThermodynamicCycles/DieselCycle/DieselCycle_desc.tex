% Verbal description for system DieselCycle (DieselCycle_desc.tex)
% Generated by MTT on Thu Dec 4 15:59:55 GMT 1997.

% %%%%%%%%%%%%%%%%%%%%%%%%%%%%%%%%%%%%%%%%%%%%%%%%%%%%%%%%%%%%%%%
% %% Version control history
% %%%%%%%%%%%%%%%%%%%%%%%%%%%%%%%%%%%%%%%%%%%%%%%%%%%%%%%%%%%%%%%
% %% $Id$
% %% $Log$
% Revision 1.1  1997/12/09  12:30:26  peterg
% Initial revision
%
% Revision 1.1  1997/12/08  09:37:04  peterg
% Initial revision
%
% %%%%%%%%%%%%%%%%%%%%%%%%%%%%%%%%%%%%%%%%%%%%%%%%%%%%%%%%%%%%%%%

   The acausal bond graph of system \textbf{DieselCycle} is
   displayed in Figure \Ref{DieselCycle_abg} and its label
   file is listed in Section \Ref{sec:DieselCycle_lbl}.
   The subsystems are listed in Section \Ref{sec:DieselCycle_sub}.


The Diesel cycle is a simple closed thermodynamic cycle with four parts:
\begin{enumerate}
\item Isentropic compression
\item Heating at constant pressure
\item Isentropic expansion
\item Cooling at constant volume
\end{enumerate}

The subsystem \textbf{Cycle} (Section \Ref{sec:Cycle}) is a two-port
component describing an ideal gas. It has two energy ports which, with
integral causality correspond to
\begin{enumerate}
\item Entropy flow in; temperature out
\item Volume rate of change in; pressure out
\end{enumerate}

In contast to the Otto cycle (see Table
\Ref{tab:cycles} where each table entry gives the causality on the
heat and work ports respectively). The ideal Diesel cycle has
derivative causality on the {\bf [Work]} port for one part of the
cycle.

To avoid this causality change, the Diesel cycle is approximated by
applying the volume change from a pressure source via a resistance
{\bf R} component. During the {\em heat injection\/} part of the
cycle, the resistance parameter $r\approx 0$, but during the other parts of
the cycle, the resistance parameter $r\approx \inf$.

The simulation parameters appear in Section
\Ref{sec:DieselCycle_numpar.txt}. The results are plotted against time
as follows:
\begin{itemize}
\item Volume (Figure \Ref{fig:DieselCycle_odeso.ps-DieselCycle-cycle-V})
\item Pressure (Figure
\Ref{fig:DieselCycle_odeso.ps-DieselCycle-cycle-P})
\item Entropy (Figure \Ref{fig:DieselCycle_odeso.ps-DieselCycle-cycle-S})
\item Temperature (Figure
\Ref{fig:DieselCycle_odeso.ps-DieselCycle-cycle-T})
\end{itemize}

These values are replotted as the standard PV and TS diagrams in
Figures
\Ref{fig:DieselCycle_odeso.ps-DieselCycle-cycle-V:DieselCycle-cycle-P}
and
\Ref{fig:DieselCycle_odeso.ps-DieselCycle-cycle-S:DieselCycle-cycle-T}
respectively.





