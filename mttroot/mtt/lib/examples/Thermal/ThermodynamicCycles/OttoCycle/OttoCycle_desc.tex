% Verbal description for system OttoCycle (OttoCycle_desc.tex)
% Generated by MTT on Thu Dec 4 15:59:55 GMT 1997.

% %%%%%%%%%%%%%%%%%%%%%%%%%%%%%%%%%%%%%%%%%%%%%%%%%%%%%%%%%%%%%%%
% %% Version control history
% %%%%%%%%%%%%%%%%%%%%%%%%%%%%%%%%%%%%%%%%%%%%%%%%%%%%%%%%%%%%%%%
% %% $Id$
% %% $Log$
% Revision 1.1  1997/12/08  09:37:04  peterg
% Initial revision
%
% %%%%%%%%%%%%%%%%%%%%%%%%%%%%%%%%%%%%%%%%%%%%%%%%%%%%%%%%%%%%%%%

   The acausal bond graph of system \textbf{OttoCycle} is
   displayed in Figure \Ref{OttoCycle_abg} and its label
   file is listed in Section \Ref{sec:OttoCycle_lbl}.
   The subsystems are listed in Section \Ref{sec:OttoCycle_sub}.


The Otto cycle is a simple closed thermodynamic cycle with four parts:
\begin{enumerate}
\item Isentropic compression
\item Heating at constant volume
\item Isentropic expansion
\item Cooling at constant volume
\end{enumerate}

The subsystem \textbf{Cycle} (Section \Ref{sec:Cycle}) is a two-port
component describing an ideal gas. It has two energy ports which, with
integral causality correspond to
\begin{enumerate}
\item Entropy flow in; temperature out
\item Volume rate of change in; pressure out
\end{enumerate}

In Bond Graph terms, each of the four parts of the Otto cycle
correspond to integral causality as in each case a \emph{flow} is
constrained. This is in contrast to other cycles listed in Table
\Ref{tab:cycles} where each table entry gives the causality on the
heat and work ports respectively. This is possibly why the Otto cycle
is conceptually and practically simple.

The simulation parameters appear in Section
\Ref{sec:OttoCycle_numpar.txt}. The results are plotted against time
as follows:
\begin{itemize}
\item Volume (Figure \Ref{fig:OttoCycle_odeso.ps-OttoCycle-cycle-V})
\item Pressure (Figure
\Ref{fig:OttoCycle_odeso.ps-OttoCycle-cycle-P})
\item Entropy (Figure \Ref{fig:OttoCycle_odeso.ps-OttoCycle-cycle-S})
\item Temperature (Figure
\Ref{fig:OttoCycle_odeso.ps-OttoCycle-cycle-T})
\end{itemize}

These values are replotted as the standard PV and TS diagrams in
Figures
\Ref{fig:OttoCycle_odeso.ps-OttoCycle-cycle-V:OttoCycle-cycle-P}
and
\Ref{fig:OttoCycle_odeso.ps-OttoCycle-cycle-S:OttoCycle-cycle-T}
respectively.





